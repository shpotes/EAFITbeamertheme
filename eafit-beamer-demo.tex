%!TEX program = xelatex

% Name           : eafit-beamer-demo.sty
% Author         : Benjamin Weiss & Santiago Hincapie
% Version        : 0.1
% Created on     : 05.05.2013
% Last Edited on : 10.05.2020
% Copyright      : Copyright (c) 2013-2014 by Benjamin Weiss. All rights reserved.
% License        : This file may be distributed and/or modified under the
%                  GNU Public License.
% Description    : EAFIT beamer theme demonstration. Also includes a short
%                  Tutorial regarding the beamer class.

\documentclass[compress]{beamer}
%--------------------------------------------------------------------------
% Common packages
%--------------------------------------------------------------------------
\usepackage[spanish]{babel}
\usepackage{graphicx}
\usepackage{multicol}
% Erweiterte Tabellenfunktionen
\usepackage{tabularx,ragged2e}
\usepackage{booktabs}
% Listingserweiterung
\usepackage{listings}
\lstset{ %
  language=[LaTeX]TeX,
  basicstyle=\normalsize\ttfamily,
  keywordstyle=,
  numbers=left,
  numberstyle=\tiny\ttfamily,
  stepnumber=1,
  showspaces=false,
  showstringspaces=false,
  showtabs=false,
  breaklines=true,
  frame=tb,
  framerule=0.5pt,
  tabsize=4,
  framexleftmargin=0.5em,
  framexrightmargin=0.5em,
  xleftmargin=0.5em,
  xrightmargin=0.5em
}

%--------------------------------------------------------------------------
% Load theme
%--------------------------------------------------------------------------
\usetheme{eafit}

%\usepackage{dtklogos} % must be loaded after theme
\usepackage{tikz}
\usetikzlibrary{mindmap,backgrounds}

%--------------------------------------------------------------------------
% General presentation settings
%--------------------------------------------------------------------------
\title{EAFIT Beamer Theme}
\subtitle{Demonstracion y breve introduccion en Beamer}
\date{Ultima actualizacion: \today}
\author{Santiago Hincapie-Potes}
\institute{Universidad EAFIT}

%--------------------------------------------------------------------------
% Notes settings
%--------------------------------------------------------------------------
\setbeameroption{show notes}

\begin{document}
% --------------------------------------------------------------------------
% Titlepage
%--------------------------------------------------------------------------

\maketitle

%--------------------------------------------------------------------------
% Table of contents
%--------------------------------------------------------------------------
\section*{Contenido}
\begin{frame}{Contenido}
	% hideallsubsections se recomienda para presentaciones más largas
	\tableofcontents[hideallsubsections]
\end{frame}

%--------------------------------------------------------------------------
% Content
%--------------------------------------------------------------------------
\section{Introduccion}

\begin{frame}{Que es Beamer?}
  Beamer es una clase de \LaTeX\ para crear diapositivas, posee una amplia gama
  de plantillas (como esta) y un conjunto interesante de características crear
  buenas presentaciones.

  Personalmente odiaba el tema oficial de la universidad, así que inspirado en
  la plantilla de la Universidad de RheinMain creé este tema, este trata de
  seguir los colores y imagen corporativa, con un poco mas de estilo.
\end{frame}

\begin{frame}{Dependencias}
	Para utilizar con exito este tema, el sistema debe cumplir
	\begin{itemize}
		\item XeLaTex debe estar instalado y ser usado para compilar el archivo
		\texttt{.tex}.
		\item Además de los paquetes estándar, deben instalarse los paquetes \
		texttt {beamer}, \ texttt {pgf} y \ texttt {xcolor}. 
		\item Las fuentes \quoted{Flama-Light}, \quoted{Flama-Book} y
		\quoted{Flama-Medium} deben instalarse. ver carpeta \texttt{fonts/}
	\end{itemize}
\end{frame}

\section{Tutorial}
\begin{frame}[containsverbatim]{Estructura basica}
La estructura basica es simple:
\begin{lstlisting}
\documentclass[compress]{beamer}
% Cargar el tema
\usetheme{eafit}
% Configuracion de la presentacion
\title{Titulo de la presentacion}
\subtitle{Subtitulo de la presentacion}
\author{Tu nombre}
\institute{Universidad EAFIT}
\begin{document}
% Diapos
\end{document}
\end{lstlisting}
\end{frame}

\begin{frame}{Colores }
  
Los colores corporativos mas estandarés fueron agregados.

\begin{multicols}{2}

\setbeamercolor{EAFITBlueDarkDemo}{fg=EAFITBlueDark,bg=white}
\begin{beamercolorbox}[wd=\linewidth,ht=2ex,dp=0.7ex]{EAFITBlueDarkDemo}
	\texttt{EAFITBlueDark}
\end{beamercolorbox}
\setbeamercolor{EAFITBlueLightDemo}{fg=EAFITBlueLight,bg=white}
\begin{beamercolorbox}[wd=\linewidth,ht=2ex,dp=0.7ex]{EAFITBlueLightDemo}
	\texttt{EAFITBlueLight}
\end{beamercolorbox}
\setbeamercolor{EAFITGreenDemo}{fg=EAFITGreen,bg=white}
\begin{beamercolorbox}[wd=\linewidth,ht=2ex,dp=0.7ex]{EAFITGreenDemo}
	\texttt{EAFITGreen}
\end{beamercolorbox}
\setbeamercolor{EAFITGrayDemo}{fg=EAFITGray,bg=white}
\begin{beamercolorbox}[wd=\linewidth,ht=2ex,leftskip=.5ex,dp=0.7ex]{EAFITGrayDemo}
	\texttt{EAFITGray}
\end{beamercolorbox}


\setbeamercolor{EAFITBlueDarkDemoBg}{fg=white,bg=EAFITBlueDark}
\begin{beamercolorbox}[wd=\linewidth,ht=2ex,leftskip=.5ex,dp=0.7ex]{EAFITBlueDarkDemoBg}
	\texttt{EAFITBlueDark}
\end{beamercolorbox}
\setbeamercolor{EAFITBlueLightDemoBg}{fg=white,bg=EAFITBlueLight}
\begin{beamercolorbox}[wd=\linewidth,ht=2ex,leftskip=.5ex,dp=0.7ex]{EAFITBlueLightDemoBg}
	\texttt{EAFITBlueLight}
\end{beamercolorbox}
\setbeamercolor{EAFITGreenDemoBg}{fg=white,bg=EAFITGreen}
\begin{beamercolorbox}[wd=\linewidth,ht=2ex,leftskip=.5ex,dp=0.7ex]{EAFITGreenDemoBg}
	\texttt{EAFITGreen}
\end{beamercolorbox}
\setbeamercolor{EAFITGrayDemoBg}{fg=white,bg=EAFITGray}
\begin{beamercolorbox}[wd=\linewidth,ht=2ex,leftskip=.5ex,dp=0.7ex]{EAFITGrayDemoBg}
	\texttt{EAFITGray}
\end{beamercolorbox}

\end{multicols}

\end{frame}


\begin{frame}[containsverbatim]{Estructura de una diapositiva}
La estructura de una presentacion de Beamer es muy similar a la de cualquier
otro documento de \LaTeX, usamos \lstinline!section! y \lstinline!subsection!
para limitar las secciones, ademas del entorno \lstinline!frame! para crear una
diapositivas
\begin{lstlisting}
\section{Primera seccion}
\subsection{primera  subseccion}
\begin{frame}
\frametitle{titulo de la diapositiva}
% Contenido de la diapositiva
\end{frame}
\end{lstlisting}
\end{frame}

\begin{frame}[containsverbatim]{Titulo y tabla de contenidos}
La pagina de titulo se crea con
\begin{lstlisting}
\maketitle
\end{lstlisting}
Y la tabla de contenido con
\begin{lstlisting}
\begin{frame}{Contenidos}
	\tableofcontents[hideallsubsections]
\end{frame}
\end{lstlisting}
La opción \lstinline!hideallsubsections! es util para presentaciones largas
\end{frame}

\subsection{Enumeraciones}
\begin{frame}[containsverbatim]{Enumeraciones}
Para enumerar tenemos disponibles los entornos \lstinline!enumerate! y \lstinline!itemize!
\begin{enumerate}
	\item Punto 1
	\item Punto 2
	\begin{itemize}
		\item Punto 1
		\item Punto 2
	\end{itemize}
	\item Punto 3
\end{enumerate}
\end{frame}

\subsection{Resaltadores}
\begin{frame}[containsverbatim]{Resaltadores}
  En Beamer la funcion \lstinline!\alert! define una alerta, es util para resaltar
  paralabras individuales:
  \begin{itemize}
	\item \alert{Texto resaltado}
  \end{itemize}
\end{frame}

\subsection{Bloques}
\begin{frame}[containsverbatim]{Bloques simples}
  Podemos usar la estructura de \lstinline!\block! para crear un bloque simple
\begin{block}{Bloque con enumeracion}
	\begin{itemize}
		\item Punto 1
		\item Punto 2
	\end{itemize}
\end{block}
\begin{lstlisting}
\begin{block}{Bloque con enumeracion}
	\begin{itemize}
		\item Punto 1
		\item Punto 2
	\end{itemize}
\end{block}
\end{lstlisting}
\end{frame}

\begin{frame}[containsverbatim]{Bloques de alerta}
  \begin{alertblock}{Bloque de alerta}
	Alerta!
  \end{alertblock}
\begin{lstlisting}
\begin{alertblock}{Alert Block}
  Alerta!
\end{alertblock}
\end{lstlisting}
\end{frame}

\begin{frame}[containsverbatim]{Bloques de ejemplo}
  \begin{exampleblock}{Bloque de ejemplo}
    Es similar a los anteriores, pero coloreado con otro color
  \end{exampleblock}
\begin{lstlisting}
\begin{exampleblock}{Bloque de ejemplo}
    Es similar a los anteriores, pero coloreado con otro color
  \end{exampleblock}
\end{lstlisting}
\end{frame}


\section{Ejemplos}
\begin{frame}{Ejemplos adicionales}
  A continuacion se mostrarán varias diapositivas de ejemplo sin ninguna
  explicacion adicional, en caso de ser necesaria por favor remitasé al codigo
  fuente de este documento.
\end{frame}
\subsection{Imagenes}
\begin{frame}{Foto con Copyright}
	\begin{figure}
		\centering
		\includegraphicscopyright[width=\linewidth]{res/photo.jpg}{Copyright by \href{http://netzlemming.deviantart.com/}{Netzlemming}, \href{http://creativecommons.org/licenses/by-nc/3.0/}{CC BY-NC 3.0 License}}
	\end{figure}
\end{frame}

\subsection{Tablas}
\begin{frame}{Tablas}
\begin{table}[]
	\caption{Selection of window function and their properties}
	\begin{tabular}[]{lrrr}
		\toprule
		\textbf{Window}			& \multicolumn{1}{c}{\textbf{First side lobe}}	
		                    & \multicolumn{1}{c}{\textbf{3\,dB bandwidth}}
		                    & \multicolumn{1}{c}{\textbf{Roll-off}} \\
		\midrule
		Rectangular		  		& 13.2\,dB	& 0.886\,Hz/bin	& 6\,dB/oct		\\[0.25em]
		Triangular		 	  	& 26.4\,dB	& 1.276\,Hz/bin	& 12\,dB/oct	\\[0.25em]
		Hann		    	  		& 31.0\,dB	& 1.442\,Hz/bin	& 18\,dB/oct	\\[0.25em]
		Hamming			  	  	& 41.0\,dB	& 1.300\,Hz/bin	& 6\,dB/oct		\\
		\bottomrule
	\end{tabular}
	\label{tab:WindowFunctions}
\end{table}
\end{frame}

\subsection{Formulas}
\begin{frame}{Formulas}
%\begin{block}{Fourierintegral}
\begin{equation*}
F(\textrm{j}\omega) = \int\limits_{-\infty}^{\infty} f(t)\cdot\textrm{e}^{-\textrm{j}\omega t} dt
\end{equation*}
%\end{block}
\end{frame}

\subsection{Notas de pie}
\begin{frame}{Notas de pie}
	Lorem ipsum dolor sit amet, consetetur sadipscing elitr, sed diam nonumy eirmod tempor invidunt ut labore et dolore magna aliquyam erat, sed diam voluptua. At vero eos et accusam et justo duo dolores et ea rebum. Stet clita kasd gubergren, no sea takimata sanctus est Lorem ipsum dolor sit amet. Lorem \footnote{Lorem ipsum dolor sit amet} ipsum dolor sit amet, consetetur sadipscing elitr, sed diam nonumy eirmod tempor invidunt ut labore et dolore magna aliquyam erat, sed diam voluptua. At vero eos et accusam et justo duo dolores et ea rebum. Stet clita kasd gubergren, no sea takimata sanctus est Lorem ipsum dolor sit amet.
\end{frame}

\subsection{Notas}
\begin{frame}{Diapositivas con notas}
  Esta diapositiva es para el publico.

  Los siguientes programas están disponibles para su visualización:
	\begin{itemize}
    \item pdfpc (Linux) \\\url{https://github.com/pdfpc/pdfpc}
		\item Splitshow (Mac OS X)\\\url{https://code.google.com/p/splitshow/}
		\item pdf-presenter (Windows)\\\url{https://code.google.com/p/pdf-presenter/}
	\end{itemize}
\end{frame}

\note{
  Esta diapositiva contiene notas sobre la presentacion

  Los siguientes programas están disponibles para su visualización:
	\begin{itemize}
    \item pdfpc (Linux) \\\url{https://github.com/pdfpc/pdfpc}
		\item Splitshow (Mac OS X)\\\url{https://code.google.com/p/splitshow/}
		\item pdf-presenter (Windows)\\\url{https://code.google.com/p/pdf-presenter/}
	\end{itemize}
}

\subsection{Columnas}
\begin{frame}{Dos Columnas}
	\begin{multicols}{2}
		Lorem ipsum dolor sit amet, consetetur sadipscing elitr, sed diam nonumy eirmod tempor invidunt ut labore et dolore magna aliquyam erat, sed diam voluptua. At vero eos et accusam et justo duo dolores et ea rebum. Stet clita kasd gubergren, no sea takimata sanctus est Lorem ipsum dolor sit amet.
		\begin{itemize}
        	\item uno
        	\item dos
		\end{itemize}
	\end{multicols}
\end{frame}

\begin{frame}{Partir columnas}
	\begin{multicols}{2}
		Lorem ipsum dolor sit amet, consetetur sadipscing elitr, sed diam nonumy eirmod tempor invidunt ut labore et dolore magna aliquyam erat, sed diam voluptua. At vero eos et accusam et justo duo dolores et ea rebum. Stet clita kasd gubergren, no sea takimata sanctus est Lorem ipsum dolor sit amet.
		\columnbreak
		\begin{itemize}
        	\item uno
        	\item dos
		\end{itemize}
	\end{multicols}
\end{frame}

\begin{frame}{Bibliografia}
	\begin{thebibliography}{10}
    
	\beamertemplatebookbibitems
	\bibitem{Oppenheim2009}
	Alan~V.~Oppenheim
	\newblock \doublequoted{Discrete-Time Signal Processing}
	\newblock Prentice Hall Press, 2009

	\beamertemplatearticlebibitems
	\bibitem{EBU2011}
	European~Broadcasting~Union
	\newblock \doublequoted{Specification of the Broadcast Wave Format (BWF)}
	\newblock 2011
  \end{thebibliography}
\end{frame}

\section{Panorama}
\begin{frame}{Bugs conocidos}
	\begin{itemize}
		\item De momento este tema es un unico archivo \texttt{sty}, lo cual es
      bastante suboptimo, debería crearse un archivo separado para fuentes,
      colores, etc.
    \item No ha sido probado ni en Windows ni en mac, es posible que genere
      errores con las fuentes.
	\end{itemize}
\end{frame}

\begin{frame}{Comentarios finales}
  Este tema es una EAFITisacion del tema HSRM creador por Benjamin Weiss. \\
  El tema se encuentra bajo \quoted{GNU Public License}, por lo tanto, este
  puede transmitirse y modificarse siempre que se mantenga la licencia.
	
	Si poseen preguntas/comentarios sobre el tema, pueden realizarlas utilizando
  un \href{https://github.com/shpotes/EAFITbeamertheme}{issue de github},
  adicionalmente pueden contactar al ``creador'' del tema por correo
  electronico, shinca12[at]eafit[dot]edu[dot]co.
\end{frame}

\end{document}
